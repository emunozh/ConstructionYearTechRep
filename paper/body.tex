\section{The Role of Construction Year of Buildings in Urban Energy Models}
 
The past years have seen a proliferation of urban models developed for the
estimation of urban energy demand
\cite{Calderon.2015,Mata.2014,MunozH.2014.IBPSA-JP,MunozH.2014.IJM,Chingcuanco.2012,Fracastoro.2011}.
Space heating has a large share (in many cities of higher latitudes: the
largest share) in urban energy demand. Space heating used to be estimated based
on urban area typologies, which offer different settlement types associated
with typical heat demands 
\cite{Roth.1980,Roth.1981,Blesl.2002,Everding.2004,Blesl.2007,Genske.2009,Erhorn.2011}.
These heat demands were based on empirical findings in case studies.\\
 
The dominant methods for estimating space heating demand today focus on the
individual building. They either use building typologies with typical energy
demand coefficients for each building type (requiring the analyst to classify
individual buildings into types) or compute heat demand at the individual
building level.\\
 
For both methods, the construction year of buildings is a variable of paramount
importance. It is the construction year which exerts the greatest influence on
the physical properties of the building envelope, and building envelope, in
turn, is the single most important determinant of heat demand in many countries
of higher latitude, as it is responsible for heat transmission losses.
Construction techniques and legal requirements on the energy efficiency of
buildings vary between construction epochs. In Germany, for example, there have
been legal requirements on the energy efficiency of newly constructed buildings
since the 1970s. They have been updated, on average, every twelve years,
setting increasingly stringent standards on the heat transmission properties of
the building shell.\\
 
Construction years are not always available though. For example, in the
electronic cadastre of the city of Hamburg (a city of 1,8 million inhabitants
and 300.00 buildings) around half of all residential buildings do not carry a
construction year. In this paper, we offer several algorithms for estimating
construction years based on existing data and illustrate the effect of these
methods. The algorithms follow a diffusion logic, aiming to replicate the
real-world processes of the past that gave rise to new construction in
different parts of the city over the years.\\

Diffusion models have been applied in urban simulation to the topic of
technology diffusion. \citeA{Linder.2013} uses a diffusion algorithm for the
simulation of the installation of photovoltaic panels and its contribution to
an increasing distributed energy supply system. \citeA{Guseo.2008}
implement a cellular automate framework for the simulation of innovation
diffusion, the authors compare their model with an aggregate Generalized Bass
Model (GBM), the author argue in favor of a disaggregated, simulation based
approach.
%(what type of technology do they look at?).
\citeA{Schmid.2008} develop an agent-based model to simulate the diffusion of
biogas plants within Switzerland. \citeA{Peters.2002} develop an agent-based
model which incorporates the diffusion of a novel sanitation technology.\\

The aim of our paper is to provide city and energy planners with a method to
gain important information for the estimation of current and future heat
demand, needed for dimensioning and managing a city's  heat supply
infrastructure.\\ 

There is a tendency towards the development of disaggregated simulation based
models for the estimation and forecast of urban energy demand
models~\cite{Balaras.2007, Kavgic.2010, Dascalaki.2010, Dascalaki.2011,
DallO.2012, Caputo.2013, Hrabovszky.2013, Kragh.2013, Singh.2013}
%
For a review of energy demand models see~\citeA{Swan.2009} and 
\citeA{Keirstead.2012}, on both papers the authors argue in favor of disaggregated
simulation models for the estimation of energy demand.\\
 
%A word on aggregation levels in models. In estimating urban heat demand, some
%authors have used urban area typologies. There is a shift towards focussing on
%the individual building.
%This micro-level has advantages ... here include the remarks on agent-based
%models. 

\section{Diffusion Mechanisms}

The use of diffusion models has been traditionally used on models developed to
represent occurring natural phenomena.
%TODO: literature & examples
The use of diffusion models has also been used on models developed for the
simulation of technology diffusion over time. 
%
A constant among diffusion models has been the concept of time. The evolution
or spatial development of a technology, idea or population.
In this paper we make use of a diffusion model to represent a static phenomena.
The idea behind the use of a diffusion model within a static framework is to
take advantage of an interaction between elements of the model.
This interaction, define as the diffusion process, is an iterative process. In
a classical definition each iteration would represent a simulation step in
time.
In the presented paper the iterative process does not represent time and is
only present as a tool to achieve a state of convergence within the model.\\

The focus on the diffusion terminology has the aim to facilitate the
understanding of the developed algorithm for the estimation of missing years.
The goal of this algorithm is the estimation of missing construction years by
selecting a similar building within a given radius, the ``know'' construction
years propagate in space until every building has a ``known'' construction
year. This propagation of years gives our algorithm the characteristics of a
diffusion model.\\

In the energy planning community the use of diffusion models are not very
common. A good example is the use of technology diffusion algorithm for the
simulation of the diffusion of PV panels in a large region~\cite{Linder.2013}.
The use of a diffusion model is also used by~\citeA{Bagchi.2013} for the
simulation of a fire spread of an electricity grid. 
\citeA{Wittmann.2007} present an agent-based model developed for the simulation
of energy supply systems under liberalized markets. Within this model the
authors define a technology diffusion model based on the individual decisions
of the model agents.
%
Many cellular automata models could be described as a diffusion algorithm, the
use of a cellular automata for the forecasting of urban growth could be
classified as diffusion models~\cite{Han.2009,Batty.1999}. \citeA{Guseo.2008}
makes use of a cellular automata model for the development of a technology
diffusion model.\\

In this paper we present a small diffusion model for the estimation of missing
construction years, the single most important parameter for the estimation of
heat consumption of individual buildings. Nonetheless, the presented model can
be used for the simulation of all kind of diffusion phenomena. An explain
before the aim of this model is not to accurately represent simulation time but
to achieve a predefined model convergence, the estimation of all building
construction years. In order to use the presented model as a dynamic model the
user would need to explicit define the handling of time steps within the
framework.\\

For the development of the model we make use of the python language, we see
python as the de-facto language for any simulation dealing with spatial
referenced objects. De model development in the python language allows us to
use powerful tools like the PostgreSQL database without much difficulty. At the
same time the language offers powerful libraries for data manipulation
(pandas) and for spatial objects (shapely). The use of the developed functions
are presented in an Ipython notebook, making the reproduction of results
extremely easy.\\

\section{The Data We Work With}
%TODO: ALKIS

In this paper we apply the developed algorithm to a preprocessed data sample
retrieved from the official digital cadastre of the city of Hamburg
ALKIS~\cite{AdV.2008}. The selected data is from year 2010 because many of our
models use the year 2010 as a basis year. The entire digital cadastre is now
open and can be freely downloaded from
daten-hamburg.de\footnote{\url{http://daten-hamburg.de/geographie_geologie_geobasisdaten/ALKIS_Liegenschaftskarte/}}.
%
The digital cadastre provides many information about the urban environment of
the city. For this analysis we only make use of the data set describing the
building stock.
%
The data set describing the building stock contains information about every
single building in the city. Each building is represented in space with an
accurate geometry of its perimeter ans some attributes like: building use,
construction type of construction year (see Table~\ref{tab:attributes}). The
latter is an attribute that is not available for every building in the
database. Table~\ref{tab:data} list the share of buildings with an unknown
construction year in the 2010 digital cadastre. 30\% of all buildings and 51\%
of all residential buildings have a known construction year. We use this
buildings as a starting point for the estimation of construction years for the
rest of the building of the digital cadastre.\\

\begin{table}[htb]
    \centering
    \caption{Some of the attributes of the individual buildings on the digital
    cadastre}\label{tab:attributes}
    \begin{tabular}{lll}
        \toprule
        \textbf{Attribute Description} & \textbf{Name} & \textbf{Data
        type} \\
        \midrule
        Unique Identifier             & UUID        & String          \\
        Construction type             & BAW         & Categorical     \\
        Building function             & GFK         & Categorical     \\
        Construction year             & BAJ         & Integer         \\
        District                      & Stadtteil   & String          \\
        Living area                   & sqm         & Float           \\
        Shell area of wall            & shell\_wall & Float           \\
        Shell area of building        & shell       & Float           \\
        %Number of performed retrofits & retrofits   & Integer         \\
        ID of k nearest neighbours    & neighbours  & Array           \\
        Statistical area              & statgeb     & String          \\
        Geometry of the building      & geometry    & Shapely Polygon \\
        Simplified geometry of the building      & simple\_geometry    & Shapely Polygon \\
        \bottomrule
    \end{tabular}
\end{table}

\begin{table}[htb]
    \centering
    \caption{Available records from the digital cadastre}\label{tab:data}
    \begin{tabular}{lrrr}
        \toprule
        \textbf{Used filter} & \textbf{\# of records} &
        \textbf{Share}& \textbf{\dots from}\\
        & & & \textbf{from total} & \textbf{residential}\\
        \midrule
                                Records on the digital cadastre. & 369~416 & 100\% & / \\
\hspace{0.2cm} \ding{253} \dots with a known construction year& 110~845 & \textbf{30\%} & / \\
\hspace{0.2cm} \ding{253} \dots residential buildings& 194~996 & 53\% & 100\%\\
\hspace{0.8cm} \ding{253} \dots with a known construction year& 98~504  & 27\% & \textbf{51\%}\\
\bottomrule
    \end{tabular}
\end{table}

The data uploaded with the model scripts is a preprocessed data from the
digital cadastre. With the available data we populate a postgreSQL database and
index the database with two columns containing geographical information. In
order to simplify and increase the simulation speed of our models we abstract
the detailed geometrical information of the individual buildings
(see~\cite{MunozH.2015.MEQ}). The used data contains only the simplified
geometrical information. The geometrical information of the buildings is used
in the model to compute the minimal distance between objects (through the
shapely package) and in the PostgreSQL database for the definition of
neighbourhoods.\\ 

In the presented paper we interpret the data set as a pandas data frame. This
is possible because the building sample is relative small with 1757 buildings.
In order to process larger amounts of data we make use of the PostgreSQL
database from which we read the needed input data and write the estimated new
data.\\

In order to allow the full reproducibility of the model we provide the used
data set as a \verb|hdf| file. This file can be directly imported as a pandas
data frame into any python working environment. Other software tools should be
able to read and interpret most of the data saved on this file, the
geometrical data of the individual buildings is stores as Well Known Text (WKT)
objects, a standard define by the Open Geospatial Consortium (OGC).
We also provide the used script to communicate to our PostgreSQL database to
generate the hdf file for information purposes.\\

\section{Structure, Rationale, Results and Suggestion for Refining the
Mechanism}

On this section we present the main components of the developed model. First we
describes the definition of neighbourhoods within the selected building sample.
We define this neighborhoods within the PostgreSQL database in order to take
advantage of the advance spatial indexing of the database. With the
neighbourhood define we present the main algorithm used for the definition of
the diffusion model, On a next step we rank each neighbour according to its
characteristics for the estimation of construction years. We end the main
section of the paper by discussing the results and proposing further
improvements for the model.\\ 

\subsection{Definition of k-nearest Neighbours}

An essential component of the diffusion model is the definition of a
neighborhood. There are many ways to define a neighbourhood. In this example we
have a simple static definition of neighbourhood. We allocate neighbours to
each element within our PostgreSQL database. The definition of neighbourhood is
created as a function of two variables. The first variable is the radius
$r$ to select possible neighbours and the second the amount of neighbours
$k$ to select. Because the definition of this neighbourhood is rather simple we
can define it within the spatial database, we make this via an SQL script. This
script is performed for each building in the database.\\

\begin{equation}\label{eq:k-nearest}
    N_{[1-k],i} = sort(B_{n, i}) 
\end{equation}

\begin{equation}\label{eq:neighbours}
    B_{n, i} = B_j \forall j: \text{within~buffer} (B_i, r)
\end{equation}
    %\text{~if $B_j$ within buffer(centroid($B_i$), $r$)} 

\noindent
\textbf{Where:}\vspace{.3cm} \\
\begin{tabular}{lp{15cm}}
    $B$   & Buildings sample\\
    $N_i$   & $k$ number of Neighbours within radius $r$ of centroid of building
    $i$\\
\end{tabular}\\ \\

\begin{figure}[htb]
    \centering
    \includegraphics[width=0.5\textwidth]{FIGURES/buffer.png}
    \caption{Example of the k-nearest algorithm with $k=10$ and a radius
    $r=100$}\label{fig:k-nearest}
\end{figure}

The developed scripts for the definition of neighbourhoods is described on
scrips \verb|spatialQNeighbours.sql|. On this scripts the radius is set to
$r=500$ meters (see code line 10) and the number of buildings, define as the
nearest $k$ neighbours is set to $k=30$ on the last code line. The SQL script
selects the first $k$ elements because the elements are sorted by euclidean
distance to the selected building for which we are defining the neighbourhood.
The variable within this SQL script is the selected building \verb|UUID|, in
this case the \verb|UUID| is set to \verb|DEHHALKA10000w1Y|.\\

A graphical representation of the neighbourhood definition is depicted on
Figure~\ref{fig:k-nearest}. In this case we define the radius $r=100$ and
selected the nearest 10 buildings $k=10$. This neighbourhood definition is
static because it is equal for every single building in the database, in the
final section of this paper we argue for a more elaborated definition of
neighborhood. In order to develop more elaborated neighbourhood definitions we
need a systematic mechanism to asses the performance of the different
definitions. For the presented paper the definition of a neighborhood is not
central as we aim to focus on the diffusion algorithm itself, a more elaborated
neighborhood definition can be implemented at any time within the presented
diffusion model.\\

\subsection{Diffusion Model in Python}

In the following section we present the developed model and discuss the
individual model components used in the model. The individual components used
in the model have been arrange as python function and each function is define
within an individual python file. This arrangement has been chosen in order to
easy the reader search for a specific model component, a disadvantage of this
arrangement is that it created some small redundancies within the model. An
implementation example of the individual components of the model is presented
as an \textbf{ipython-notebook} file. The file can be visualize within the
github repository under:
\url{https://github.com/emunozh/ConstructionYearTechRep/blob/master/Main.ipynb}
but the reader wont be able to run the file. In order to run the file and the
individual scripts the reader will have to download the repository and
configure a python environment with the required libraries listed on the home
page of the repository. We also provide the small data-set used in this paper
as a hdf (high density format) file. The file can be directly imported as a
pandas dataframe.\\

The algorithm will estimate the construction year for all buildings in a
database without a known construction year. The algorithm iterates through the
data set until: (1) all buildings have a know construction year; or (2) the
number of performed iterations reaches the maximum predefined number of
iterations (variable \verb|MAX_ITERATIONS|) defined in the global scope of the
script; or (3) there is no diffusion in the iteration.\\

The python function \verb|_getByear| defines the main iteration loop through
the selected building sample for the estimation of all missing construction
years, a simplified version of this function is presented on
file \verb|_getByear.py|. This function takes a \textit{Pandas} data frame
as input. For each building with an unknown construction year the loop
calls a second python function: \verb|_estimateByear|. The python function
\verb|getNeighbours| selects the predefined neighbours from the pandas data
frame. The simplest implementation of this function is a subsample of the data
frame by index, the function can pass SQL commands to the spatial database in
the case of missing neighbours of the pandas data frame. The implementation of
this function is not further discuss on this paper.\\

In this paper we present two functions than can be used as the
\verb|_estimateByear| function depending on how the libraries are imported into
the \verb|_getByear| script. The two functions use: (1) a simple median
computation of all known construction years of the building neighbours and
(2) a ranking algorithm based on characteristics of the neighbours.\\

The first python function, called \verb|_estimateByearMedian| is presented on
the python file with the same name. The function takes two parameters as
input: (1) a row of the original building sample (a row of the pandas data
frame) i.e.\ an individual building; and (2) the entire data sample, needed to
fetch the predefined neighbours. This script will simply take the median from
all neighbours with a know construction year as the estimated construction
year.
%
A possible extension to this function would be to estimate the construction
year as a weighted median based on distance and characteristics of the
neighbours.
%
We expand the algorithm into a ranking algorithm that will take into account
distance and characteristics of the buildings but does not use a madian for the
estimation of construction year.
%
The biggest problem with this method is that the algorithm assumes a dominance
of the neighbourhood for the estimation of the unknown construction year. In
the presented graphical representation of the neighbourhood definition
(Figure~\ref{fig:k.neares}) it is clear to see to which building ensemble the
selected building corresponds, an estimation of its construction year would
clearly be wrong. We aim to estimate the construction year of all buildings by
selecting the closest and most similar building to the selected one.
\\ 

\subsection{A Ranking Algorithm}

In order to define which building the most similar building is, we use the
available building attributes from the digital cadastre to rank them based on
the absolute difference to the selected building.
%
The available and used characteristics from the digital cadastre are listed on
Table~\ref{tab:attributes}. Below we make a small description of the used
attributes and comment on how the data is interpret in order to rank the
neighbours.
%
The implemented code is presented on File \verb|_getNeighbours.py|.
\\

\begin{table}[htb]
    \centering
    \caption{Attributes of the individual buildings used for the clustering of
    buildings}\label{tab:attributes}
    \begin{tabular}{llll}
        \toprule
        \textbf{Attribute Description} & \textbf{Name} & \textbf{Data
        type} & \textbf{Used}\\
        \midrule
        Construction type             & BAW         & Categorical     & * as dummy variable\\
        Building function             & GFK         & Categorical     & * as dummy variable\\
        ID of k nearest neighbours    & neighbours  & Array           & * euclidean distance\\
        Geometry of the building      & geometry    & Shapely Polygon & * euclidean distance\\
        \bottomrule
    \end{tabular}
\end{table}

\begin{description}
    \item[BAW cl: 11--18] The building construction type is used to: (1)
        distinguish between residential and non residential buildings, as non
        residential buildings do not have a define BAW;\ and (2) differentiate
        between different construction types of residential buildings.
        If the selected building has a BAW the distance value can take three
        possible values: (1) $=1$ if the neighbour building does not have a BAW
        value (is not residential); (2) $=0.5$ if the BAW is different to the
        selected buildings; and (3) $=0$ if the BAW values are the same.
        If the selected building does not have a BAW value the distance is
        equal to 1 if the neighbour has a BAW value and 0 otherwise. See
        Equation~\ref{eq:baw}.

\begin{equation}\label{eq:baw}
N_{i,j,baw} = 
  \begin{cases}
      \text{if } B_{i,baw} = \emptyset & 
      \begin{cases}
          0     & \text{if } N_{i,j,baw} = \emptyset\\
          1     & \text{else }
      \end{cases}\\
      \text{else } &
      \begin{cases}
          0 	& \text{if } B_{i,baw} = N_{i,j,baw}\\
          1     & \text{else if } N_{i,j,baw} = \emptyset\\
          0.5 	& \text{else}\\
      \end{cases}\\
  \end{cases}
\end{equation}

    \item[GKF cl: 19--21] The building function is interpreted as a boolean
        variable. If the building function is the same the distance value is
        set to 0 otherwise the value is set to 1. See Equation~\ref{eq:gfk}.
        
\begin{equation}\label{eq:gfk}
N_{i,j,gfk} = 
  \begin{cases}
      0 & \text{if } B_{i,gfk} = N_{i,j,gfk} \,
      1 & \text{else } \\
  \end{cases}
\end{equation}

    \item[SQM cl: 22--24] The floor space distance is expressed as the
        percentage absolute distance between the selected and the neighbouring
        buildings. See Equation~\ref{eq:sqm}.

\begin{equation}\label{eq:sqm}
    N_{i,j,sqm} = \left| B_{i,sqm} - N_{i,j,sqm} \right|
\end{equation}

    \item[Shell Area cl: 25--28] Analogue to the floor space the building shell
        area distance is computed as the percentage absolute distance between
        buildings. See Equation~\ref{eq.shell}.

\begin{equation}\label{eq:shell}
    N_{i,j,shell} = \left| B_{i,shell} - N_{i,j,shell} \right|
\end{equation}

    \item[Euclidian Distance cl: 29--32] The euclidean distance between
        building is computed internally by the shapely library and is computed
        as the minimal distance between geometries. See Equation~\ref{eq:diss}.

\begin{equation}\label{eq:diss}
    N_{i,j,dis} = distance(B_{i,dis} - N_{i,j,dis})
\end{equation}

\end{description}

%\textbf{Where:}\vspace{.3cm} \\
%\begin{tabular}{lp{15cm}}
%    $B$   & Buildings sample\\
%    $N_i$   & $k$ number of Neighbours within radius $r$ of centroid of building
%    $i$\\
%\end{tabular}\\

%\lstinputlisting[language=Python,
%                 style=Python,
%				 caption = Function to get the neighbours out of the data-set
%                 and compute the rank of each neighbour,
%				 label=lst:_getNeighbours]{CODE/scripts/_getNeighbours.py_}
%                 \vspace{0.8cm}

With the computed individual distance measures from the building attributes we
compute a rank for each neighbour $j$ of building $i$. The rank is computes as
the percentage sum of all the computed distances, the mathematical expression
of the computed rank is presented on Equation~\ref{eq:rank}.

\begin{equation}\label{eq:rank}
    rank_{i,j} = \sum_k N_{i,j,k} \div \max \left( \sum_k N_{i,j,k} \right)
\end{equation}

An example of the ranking algorithm is depicted on Figure~\ref{fig:rank}. The
figure shows the ranking of each predefine neighbour for a random selected
building of the sample. Each building is ranked on a scale from 1 to 0. The
lower the value the higher the similitude to the selected building.
%
The corresponding values for this neighborhood are listed on
Table~\ref{tab:rank}.
%
The ranking
algorithm does not differentiate between attributes, each attribute has the
same weight to the final ranking of the neighbour. On a more elaborated
algorithm the weighting of attributes could be modified globally of dynamically
depending of the characteristics of context of the selected building or based
of available training data. Similar to the definition of the neighbourhood such
a elaborated algorithm can only be assessed within a systematic comparison of
different implementations. The implementation of the ranking computation
occurs within the neighbourhood definition (see
File~\verb|_getNeighbours.py|), the function using the computed rank called
\verb|_estimateByearRank| is presented on
File~\verb|lst:_estimateByearRank.py|.
%
The algorithm will simple take the minimum value and define its construction
year from the selected neighbour. Within this algorithm we could define a
maximum allowed ranking. This value could be estimated by statistical means
based on a robust sample with known construction years or computed through a
calibration method with a predefined training data set.
%
Notice that the building geometry depicted on Figure~\ref{fig:rank} is a
simplified building geometry. This simplification process is described in
detail by~\citeA{MunozH.2015.MEQ}.\\

\begin{figure}[htpb]
    \centering
    \includegraphics[width=0.5\textwidth]{FIGURES/buildings_neighbour_rank.png}
    \caption{Ranking of neighbours based on building characteristics and
    euclidean distance}\label{fig:rank}
\end{figure}

\begin{table}[htb]
    \caption{Neighbours rank based on building characteristics and euclidean
    distance}\label{tab:rank}
    \input{TABLES/neighbours.tex}
\end{table}

%The missing construction years of the selected urban area are estimated based
%on the known construction years of similar neighboring buildings. A similarity
%rank is computed for the predefined neighbors (see Listing~\ref{}) based on
%known building attributes. In this example we weight every available
%characteristic equally, but a weighted ranking is possible with the same
%approach. A systematic analysis of the algorithm performance is needed in order
%to asses if a weighing of building attributes could increase the estimation
%accuracy of the algorithm. On a more advance approach the weighing of building
%attributes (and other algorithm variables, like neighborhood radius and number
%of neighbors) could be dynamically defined, defined based on a training data
%set for the specific urban area.\\
%

%\lstinputlisting[language=Python,
%                 style=Python,
%				 caption = Simplified function for the estimation of
%                 construction year based on the construction years of the
%                 predefined neighbours using a precomputed rank,
%				 label=lst:_estimateByearRank]{CODE/scripts/_estimateByearRank.py_}
%                 \vspace{0.8cm}

This process does not find a solution for all buildings, depending on the
parameters defining the neighborhood a building might not have any suitable
building within this neighborhood in order to estimate it's construction year.
In order to deal with this issue the algorithm has to run a couple of times so
that all buildings have a known construction year. On every iteration the
original data-set is updated and all estimated construction years are
interpreted as known construction years. This iterative process emulates a
diffusion process, isolated buildings will eventually get a neighbor with
similar characteristics in order to estimate it's construction year.\\

The computed results are presented in this section. The main results are
presented with the help of two figures: (1) a figure showing the diffusion
process by differentiating the known and unknown construction years of the
building sample (see Figure~\ref{fig:buildings}) and (2) showing the estimated
construction years for of each iteration (see Figure~\ref{fig:buildings_cy}).\\

On this example the algorithm can't estimate a construction year for each
building. This is because the neighborhood definition is performed for the
entire city, leaving the building at the edge of the sample without a
neighbourhood. The implementation of this algorithm only looks for neighbouring
buildings within the preselected sample, the implemented algorithm within an
under developing agent based model will fetch the missing buildings from the
PostgreSQL database. There might also be problems with the neighborhood
definition. The two variables, $r$ and $k$, are set globally. This is clearly
not ideal, a proper radius should be set based on the urban context, for more
compact urban setting a small radius could be good but won't be enough to
define a good neighborhood on a less dense urban area. Even within homogeneous
urban areas we find large building for which the algorithm might not find any
neighbour because they are larger than the predefined radius.\\

\begin{figure}[htpb]
    \centering
    \begin{tabular}{cc}
        \includegraphics[width=0.5\textwidth]{FIGURES/buildings_0.png}&
        \includegraphics[width=0.5\textwidth]{FIGURES/buildings_1.png}\\
        \includegraphics[width=0.5\textwidth]{FIGURES/buildings_2.png}&
        \includegraphics[width=0.5\textwidth]{FIGURES/buildings_6.png}\\
    \end{tabular}
    \caption{Diffusion process of estimated years for iteration 0, 1, 2 and 6.
        Unknown construction years in red}\label{fig:buildings}
\end{figure}

Figure~\ref{fig:buildings} shows clearly the diffusion process within the
building sample. On iteration $iter=0$ (initial state) only a handful of
buildings have a known construction year (black). After the first iteration
most of the buildings have a know construction year and after 6 iterations the
algorithm can't find any more construction years for the remaining buildings.
The algorithm is not able to estimate the construction year for every single
building because of the initial definition of neighborhood. It is clear that a
better definition of neighbours needs to be develop in order to improve the
performance of the algorithm.\\ 

%\begin{figure}[htpb]
%    \centering
%    \begin{tabular}{cc}
%        \includegraphics[width=0.5\textwidth]{FIGURES/buildings_0.png}&
%        \includegraphics[width=0.5\textwidth]{FIGURES/b_buildings_1.png}\\
%        \includegraphics[width=0.5\textwidth]{FIGURES/b_buildings_2.png}&
%        \includegraphics[width=0.5\textwidth]{FIGURES/b_buildings_3.png}\\
%    \end{tabular}
%    \caption{Initial state (iter=0) of selected urban area}\label{fig:buildings}
%\end{figure}

\begin{figure}[htpb]
    \centering
    \begin{tabular}{cc}
        \includegraphics[width=0.5\textwidth]{FIGURES/buildings_constructionYear_0.png}&
        \includegraphics[width=0.5\textwidth]{FIGURES/buildings_constructionYear_1.png}\\
        \includegraphics[width=0.5\textwidth]{FIGURES/buildings_constructionYear_2.png}&
        \includegraphics[width=0.5\textwidth]{FIGURES/buildings_constructionYear_6.png}\\
    \end{tabular}
    \caption{Estimated construction years on iteration 0 (initial state), 1, 2,
    and 6 (final state)}\label{fig:buildings_cy}
\end{figure}

Figure~\ref{fig:buildings_cy} shows the estimated construction years for the
same iterations as the previous figure. The diffusion process only takes
distance between buildings and attributes as parameters to decide the diffusion
direction. Available data of the urban context like streets should be included
in the algorithm to decide the diffusion direction. On the east side of the
building sample we can clearly identify a red section of buildings. The urban
fabric of this section of the city is not constructed along a north-south axis
but rather on a east-west axis. A island of buildings arranged from north to
south is rather unlikely for this particular urban setting. The introduction of
street data into the algorithm could help with this inconsistencies.\\

\subsection{On How to Improve the Mechanism}
%TODO: Expand

The introduction of streets as barriers between neighbourhoods could be used
for a more accurate definition of neighbourhood. In the same line as the
previous conclusions we argue for a more elaborate definition of
neighbourhoods. The architecture of the diffusion model proved to be
straightforward and can be easily improved. In the other hand the definition
of neighbourhoods requires more attention and the processing of more data and
the construction of algorithms capable to deal with complex data
interactions.\\

Another point to improve the model is the definition of a ranking limit. If the
computed rank isn't below a globally (or locally) define threshold the building
won't be taken into account. In this paper we did not define such a threshold
because of the following two reasons: (1) we see the use of this type of
parameters attached to some kind of model calibration, we need to define a
robust method to calibrate this type of parameters; and (2) the definition of a
ranking limit won't work with the current ranking definition because the
ranking is made at a neighbourhood level rather than globally. In order to set
a ranking limit the model needs to be able to benchmark the computed ranks to a
predefined level.\\

We want to take advantage of the rich available spatial data from the Hamburg
digital cadastre and start connecting the already available urban objects from
the database. The use of streets, urban transport networks, green areas as well
as water bodies can be integrated for a more accurate definition of
neighborhoods. An accurate definition of neighborhood will allow us to simulate
all kind of urban diffusion processes.\\

The presented results show a bias towards new constructed buildings. This is
because new buildings are more likely to have a known construction year that
older buildings are. We need to develop a mechanism to control for this bias. A
possibility it to predefine a construction year distribution at an aggregate
level and stochastically estimate the construction year of buildings with a
monte-carlo method until the prior distribution is met varying a construction
year weighting parameter.\\

\section{Outlook: Further Refinements and Applications}
%TODO: Expand

In this paper we present a diffusion model developed for the estimation of
unknown construction year of buildings described geometrically in space. We
make use of well known python libraries for the definition of the functions
used within the model. This model takes a predefine neighborhood (defines
within a PostgreSQL database) and ranks each building in the neighbourhood
based on the minimum euclidean distance to the building with an unknown
construction year as well as the difference between buildings attributes.\\

The estimation process is repeated iteratively until each building in the
database has a known construction year. This process gives the model a
diffusion character. We see the development of this model as a first step
towards more elaborate urban models able to: (1) appropriately represent
neighbourhoods for the simulation of diffusion processes and (2) a model able
to estimate missing data from spatial databases, needed for the estimation of
energy demand at a low level of aggregation.\\

We aim to improve this method by developing an algorithm able to weight the
individual characteristics of the buildings in a dynamic fashion based on
contextual information of specific urban areas, both available at a micro level
or aggregated at a statistical area, or available statistics of the national
building stock.\\

We also aim to further develop the definition of neighborhoods. We aim to
develop an algorithm for the construction of neighborhoods, able to define
dynamic neighborhoods with: (a) a clustering algorithm or (b) rich contextual
spatial elements. The use of a clustering algorithm would allow us to construct
more homogenic neighborhoods.\\

The interaction between models working a different level of scale is also an
interesting path to follow. As mentioned above, the use of predefined
distributions available at an aggregate level as simulation benchmarks is an
interesting path to follow not only for a static estimation of missing data but
for the projection of simulations. \citeA{MunozH.2015.IBPSA.Pop} project the
retrofit of the building stock at an aggregate level and use these projections
at benchmarks for the simulation of heat demand at a micro level. In a similar
fashion simulation results from a diffusion model could be benchmarked to
aggregated distributions.\\

\section{Appendix:}
 
See \url{https://github.com/emunozh/ConstructionYearTechRep}
