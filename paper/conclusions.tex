\section{Outlook: Further Refinements and Applications}
%TODO: Expand

In this paper we present a diffusion model developed for the estimation of
unknown construction year of buildings described geometrically in space. We
make use of well known python libraries for the definition of the functions
used within the model. This model takes a predefine neighborhood (defines
within a PostgreSQL database) and ranks each building in the neighbourhood
based on the minimum euclidean distance to the building with an unknown
construction year as well as the difference between buildings attributes.\\

The estimation process is repeated iteratively until each building in the
database has a known construction year. This process gives the model a
diffusion character. We see the development of this model as a first step
towards more elaborate urban models able to: (1) appropriately represent
neighbourhoods for the simulation of diffusion processes and (2) a model able
to estimate missing data from spatial databases, needed for the estimation of
energy demand at a low level of aggregation.\\

We aim to improve this method by developing an algorithm able to weight the
individual characteristics of the buildings in a dynamic fashion based on
contextual information of specific urban areas, both available at a micro level
or aggregated at a statistical area, or available statistics of the national
building stock.\\

We also aim to further develop the definition of neighborhoods. We aim to
develop an algorithm for the construction of neighborhoods, able to define
dynamic neighborhoods with: (a) a clustering algorithm or (b) rich contextual
spatial elements. The use of a clustering algorithm would allow us to construct
more homogenic neighborhoods.\\

The interaction between models working a different level of scale is also an
interesting path to follow. As mentioned above, the use of predefined
distributions available at an aggregate level as simulation benchmarks is an
interesting path to follow not only for a static estimation of missing data but
for the projection of simulations. \citet{MunozH.2015.IBPSA.Pop} project the
retrofit of the building stock at an aggregate level and use these projections
at benchmarks for the simulation of heat demand at a micro level. In a similar
fashion simulation results from a diffusion model could be benchmarked to
aggregated distributions.\\

\section{Appendix:}

See \url{https://github.com/emunozh/ConstructionYearTechRep} for the entire
repository.\\ 

You can clone the repository to run in locally via:\\
\verb|git clone https://github.com/emunozh/ConstructionYearTechRep.git|\\

The \textbf{mechanise} description can be read directly on github under:\\
\url{https://github.com/emunozh/ConstructionYearTechRep/blob/master/Main.ipynb}
\\
or via nbviewer under:\\
\url{http://nbviewer.ipython.org/github/emunozh/ConstructionYearTechRep/blob/master/Main.ipynb}.\\

The \textbf{performance} of the mechanism can be read directly on github
under:\\
\url{https://github.com/emunozh/ConstructionYearTechRep/blob/master/Performance.ipynb}
\\
or via nbviewer under:\\
\url{http://nbviewer.ipython.org/github/emunozh/ConstructionYearTechRep/blob/master/Performance.ipynb}


The \textbf{OLS} performance can be read directly on github
under:\\
\url{https://github.com/emunozh/ConstructionYearTechRep/blob/master/Performance OLS.ipynb}
\\
or via nbviewer under:\\
\url{http://nbviewer.ipython.org/github/emunozh/ConstructionYearTechRep/blob/master/Performance OLS.ipynb}


